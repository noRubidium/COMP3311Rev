\documentclass[a4paper]{scrartcl}
\usepackage[l2tabu,orthodox]{nag}% Old habits die hard. All the same, there are commands, classes and packages which are outdated and superseded. nag provides routines to warn the user about the use of those.

\usepackage[all,error]{onlyamsmath}% Error on deprecated math commands like $$ $$.
\usepackage{ amssymb }
\usepackage{listings}
\title{COMP3311}
\lstset{language=SQL}
\author{Minjie Shen}

\newcommand{\set}[2]{\{#1|#2\}}

\begin{document}
  \maketitle

\section{Data models}
  \subsection{Entity-Relationship Model}
    \subsubsection{definition}
      The Entity-Relationship (ER) model is a highlevel conceptual data model\\
    \subsubsection{Entity type}
      Group of object with the same properties (similar to type in other programming languages)
    \subsubsection{Attribute}
      A porperty of object (entity), describe properties of entities.
    \subsubsection{Relationship}
      n-way, non-directional.
  \subsection{Entity and Attributes}
    Mathematically, an attribute A of an entity type E is a function, $A:~E \rightarrow P(V)$\\
    For single-valued attributes, A(e) must be a singleton.
    \subsubsection{key}
      a set of attributes that uniquely identifies an entity.
    \subsubsection{syntax}
      We can describe schemata with composite attributes using ()’s and with multi-valued attributes using \{\}’s.
  \subsection{relationship}
    An association between things. A relationship type $R$ involving $E_1,E_2...E_n$ has property: $R \subset E_1 \times .. \times E_n$.
    \subsubsection{Recursive relation}
      An entity plays different roles in relationship.
  \subsection{Weak entity types}
    Entities that don't have key of its own
  \subsection{EER diagram}
    Specialisation: the process of defining a set of subclasses of an entity type; this entity type is called the superclass of the specialization.
    \\
    Generalisation: a reverse of previous section.
    \subsubsection{Symbol}
    $[p] - (d) ->- [s], partial disjoint$\\
    $[p] = (d) ->- [s], total disjoint$\\
    $[p] - (o) ->- [s], partial overlapping$\\
    $[p] = (o) ->- [s], total overlapping$\\
  \subsection{Relational Data Model}
    \subsubsection{Structures}
      \begin{itemize}
        \item Everything is described using relations
        \item A relation can be thought of as a named table
        \item Each column of the table corresponds to a named attribute.
        \item The set of allowed values for an attribute is called its domain.
        \item Each row is a relation
        \item Composite and multivalued attributes are disallowed!  
      \end{itemize}
    \subsubsection{Super key}
      A superkey is a set of attributes that uniquely determines a tuple.
    \subsubsection{Candidate key}
      A candidate key is a superkey, none of whose proper subsets is a superkey
    \subsubsection{Primary key}
      A primary key is a designated candidate key
    \subsubsection{Integrity constraints}
      \begin{itemize}
        \item Key constraint: candidate key value must be unique
        \item Entity integrity: attribute that's part of primary key os not nullable
        \item Referential integrity: The value of FK must occur in the other relation or be entirely
NULL.
      \end{itemize}
  \subsection{ER to RD mapping}
    \begin{itemize}
      \item For all normal entity, create R with all simple attributes and choose one of the keys as primary key for the relation
      \item For all specialised entity type of P, createa relation R with key of P and simple attributes of E.
      \item For each weak entity W, with owner type E, create R with all simple attr of W and foreign key to prime attr of E. Key will be foreign key + partial key of W.
      \item handle 1 to 1 relation. Put it in one side involved.
      \item handle 1:N relation. Put it it in the N side
      \item handle N:M relation: Create new relation and add two sides as foreign key. Key will be Key of both sides combined.
      \item handle multivalued attribute. Create new relation for it + primary key of the owner. Key will be all attributes.
      \item handle n-ary relation: same as N to M.
    \end{itemize}
\section{Relational Algebra}
  Relational Algebra is a procedural DML
  \subsection{SELECT}
    \subsubsection{definition}
      Selects a subset of the tuples of a relation r, satisfying some condition: $\sigma_B (r) = \{ t \in r | B(t) \}$
    \subsubsection{properties}
      \begin{itemize}
        \item Commutative: $\sigma_{c1}(\sigma_{c2}(R)) = \sigma_{c2}(\sigma_{c1}(R))$
        \item Combination: $\sigma_{c1}(\sigma_{c2}(R)) = \sigma_{c1~AND~c2}(R)$
      \end{itemize}
  \subsection{PROJECT}
    \subsubsection{definition}
      Selects a subset of attributes of a relation. $\pi_X(r) = \{t[X] | t \in r\}$
    \subsubsection{properties}
      \begin{itemize}
        \item if $list1 \subset list2$ then $\pi_{list1}(\pi_{list2}(R)) = \pi_{list1}(R)$, otherwise illegal.
        \item commutative: $\pi_{X}(\sigma_{B}(R)) = \sigma_{B}(\pi_{X}(R))$ if B specifies within X.
      \end{itemize}
  \subsection{UNION}
    Just normal union stuff. $r \cup s = \set{t}{t \in r~OR~t \in s }$
  \subsection{INTERSECTION}
    Just normal intersection stuff. $r \cap s = \set{t}{t \in r~AND~t \in s }$
  \subsection{DIFFERENCE}
    $r - s = \set{t}{t \in r~AND~ t \notin s}$
  \subsection{CARTESIAN PRODUCT}
    $r \times s = \set{t_1 || t_2}{t_1 \in r~AND~t_2 \in s}$\\
    equi-join: join performed based on the equality of matching attributes.
  \subsection{JOIN}
    $r \bowtie_B s = \set{t_1 || t_2}{t_1 \in r~AND~t_2 \in s ~AND~B}$\\
    \subsubsection{Equi-join}
      Equi-join is a theta-join where each comparison operator is “=”.
    \subsubsection{Natural join}
      Natural join is an equi-join where only one attribute from each comparison is retained.
  \subsection{DIVIDE}
    $R \div S = \set{t}{t \times S \in R}$
    
\section{Database Languages}
  \subsection{SQL}
    \begin{itemize}
      \item Quotes are escaped by doubling them ('')
      \item Some of the frequently-used keywords: ALTER AND CREATE FROM INSERT NOT OR SELECT TABLE WHERE
      \item Built-in datatype: strings, numbers, dates, bit-strings
      \item The NULL value is a member of all data types.
      \item Two kinds of string are available: CHAR(n) - has padding, VARCHAR(n) - no padding
      \item trailing space is ignored in comparison
      \item String
      \begin{itemize}
        \item compare: dict order, LIKE: $\%$ same as *, \_ same as .
        \item manipulation: concat: $||$, LENGTH(s): gives length, SUBSTR(str, start, length) match starting from startth char to the length (start starts from 1)
      \end{itemize}
      \item Dates
      \begin{itemize}
        \item format: DD-Mon-YYYY
        \item (start1, end1) OVERLAPS (start2, end2) -> gives whether two dateranges overlaps
      \end{itemize}
      \item Numbers
      \begin{itemize}
        \item smallint, int, bigint ... 2-bytes, 4-bytes and 8-bytes integers
        \item real, double precision... 4-bytes and 8-bytes floating point
        \item numeric(precision, scale) - The precision of a numeric is the total count of significant digits
in the whole number
        \item $\pi_X(R)$ is implemented in SQL as: SELECT X FROM R
        \item One could eliminate the duplicates by using DISTINCT.
      \end{itemize}
      \item R1 UNION R2, R1 INTERSECT R2, R1 EXCEPT R2
      \item Aggregation: SUM, AVG, MIN, MAX, COUNT
      \item Use HAVING to eliminate groups
    \end{itemize}
  \subsection{PLpgSQL}
    \subsubsection{Create type}
      CREATE TYPE, CREATE DOMAIN \\
      create domain UnswCourseCode as text \\
      check ( value ~ '[A - Z ]{4}[0 -9]{4} ' );
    %TODO: More PLpgSQL
    \subsubsection{Bored let's do it next time}
\section{Functional Dependency}
  \subsection{definition}
    X (functionally) determines Y, $X \rightarrow Y$, iff $t_1[X] = t_2[X] implies t_1[Y] = t_2[Y]$.
  \subsection{infer}
    $F infers X \rightarrow Y ( F |= X \rightarrow Y)$ means $R~satisfies~F -> R~satisfies~X\rightarrow Y$.
  \subsection{Armstrong's axioms}
    \begin{itemize}
      \item Reflexivity: $Y \subseteq X \Rightarrow X \rightarrow Y$
      \item Augmentation $\{X \rightarrow Y\} |= XZ \rightarrow YZ$
      \item Transitivity $\{X \rightarrow Y, Y \rightarrow Z\} |= X \rightarrow Z$
      \item Additivity: $\{X \rightarrow Y, X \rightarrow Z\} |= X \rightarrow YZ$
      \item Projectivity: $\{X \rightarrow YZ\} |= X \rightarrow Y$
      \item Pseudotransitivity $\{X \rightarrow Y, YZ \rightarrow W\} |= XZ \rightarrow W$
    \end{itemize}
  \subsection{Checking FD}
    $F^+$ is the smallest set of FD that contains F and closed under Armstrong's axioms.\\
    Somehow we can prove: $X^+ = \cup_{\forall X \rightarrow A \in F^+}A$ and $X\rightarrow Y \in F^+ \rightleftharpoons Y \subseteq X^+$.
  \subsection{Compute Candidate key}
    Find X such that $X^+ = R$.
    \subsubsection{Algorithm}
      Assign $X$ a super key in $F$ (all attrs on left hand side of FDs). Remove attr from $X$ while remain $X^+ = R$ till no reduction on $X$.
\section{Normal Form}
  \subsection{1NF}
    No multivalue, composite attributes.
    \subsubsection{drawback}
      A lot of duplicated entries
  \subsection{2NF}
    A prime attribute is one that's part of a candidate key. In an FD $X\rightarrow Y$, Y is fully FD on X if no $Z \subset X$ such that $Z \rightarrow Y$, otherwise it's not fulling dependent. 2NF -> all non-prime attr are fully functionally dependent. 
    \subsubsection{drawback}
      Record A and record B both associated with something. That might be stored twice.
  \subsection{3NF}
    $X \rightarrow Z$ and $Z \rightarrow Y$ and $Z \nrightarrow X$. 3NF: All NONE\_TRIVIAL FD $X \rightarrow A$ either $X$ is super key or $A$ is prime attr.
  \subsection{BCNF}
    whenever $X \rightarrow A$ holds is non-trivial, $X$ is a super key.
    http://www.cse.unsw.edu.au/~cs3311/17s1/lectures/w5/3311_week5_2.pdf
\end{document}









