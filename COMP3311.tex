\documentclass[a4paper]{scrartcl}
\usepackage[l2tabu,orthodox]{nag}% Old habits die hard. All the same, there are commands, classes and packages which are outdated and superseded. nag provides routines to warn the user about the use of those.

\usepackage[all,error]{onlyamsmath}% Error on deprecated math commands like $$ $$.
\usepackage{ amssymb }
\usepackage{listings}
\title{COMP3311}
\lstset{language=SQL}
\author{Minjie Shen}

\newcommand{\set}[2]{\{#1|#2\}}

\begin{document}
  \maketitle

\section{Data models}
  \subsection{Entity-Relationship Model}
    \subsubsection{definition}
      The Entity-Relationship (ER) model is a highlevel conceptual data model\\
    \subsubsection{Entity type}
      Group of object with the same properties (similar to type in other programming languages)
    \subsubsection{Attribute}
      A porperty of object (entity), describe properties of entities.
    \subsubsection{Relationship}
      n-way, non-directional.
  \subsection{Entity and Attributes}
    Mathematically, an attribute A of an entity type E is a function, $A:~E \rightarrow P(V)$\\
    For single-valued attributes, A(e) must be a singleton.
    \subsubsection{key}
      a set of attributes that uniquely identifies an entity.
    \subsubsection{syntax}
      We can describe schemata with composite attributes using ()’s and with multi-valued attributes using \{\}’s.
  \subsection{relationship}
    An association between things. A relationship type $R$ involving $E_1,E_2...E_n$ has property: $R \subset E_1 \times .. \times E_n$.
    \subsubsection{Recursive relation}
      An entity plays different roles in relationship.
  \subsection{Weak entity types}
    Entities that don't have key of its own
  \subsection{EER diagram}
    Specialisation: the process of defining a set of subclasses of an entity type; this entity type is called the superclass of the specialization.
    \\
    Generalisation: a reverse of previous section.
    \subsubsection{Symbol}
    $[p] - (d) ->- [s], partial disjoint$\\
    $[p] = (d) ->- [s], total disjoint$\\
    $[p] - (o) ->- [s], partial overlapping$\\
    $[p] = (o) ->- [s], total overlapping$\\
  \subsection{Relational Data Model}
    \subsubsection{Structures}
      \begin{itemize}
        \item Everything is described using relations
        \item A relation can be thought of as a named table
        \item Each column of the table corresponds to a named attribute.
        \item The set of allowed values for an attribute is called its domain.
        \item Each row is a relation
        \item Composite and multivalued attributes are disallowed!
      \end{itemize}
    \subsubsection{Super key}
      A superkey is a set of attributes that uniquely determines a tuple.
    \subsubsection{Candidate key}
      A candidate key is a superkey, none of whose proper subsets is a superkey
    \subsubsection{Primary key}
      A primary key is a designated candidate key
    \subsubsection{Integrity constraints}
      \begin{itemize}
        \item Key constraint: candidate key value must be unique
        \item Entity integrity: attribute that's part of primary key os not nullable
        \item Referential integrity: The value of FK must occur in the other relation or be entirely
NULL.
      \end{itemize}
  \subsection{ER to RD mapping}
    \begin{itemize}
      \item For all normal entity, create R with all simple attributes and choose one of the keys as primary key for the relation
      \item For all specialised entity type of P, createa relation R with key of P and simple attributes of E.
      \item For each weak entity W, with owner type E, create R with all simple attr of W and foreign key to prime attr of E. Key will be foreign key + partial key of W.
      \item handle 1 to 1 relation. Put it in one side involved.
      \item handle 1:N relation. Put it it in the N side
      \item handle N:M relation: Create new relation and add two sides as foreign key. Key will be Key of both sides combined.
      \item handle multivalued attribute. Create new relation for it + primary key of the owner. Key will be all attributes.
      \item handle n-ary relation: same as N to M.
    \end{itemize}
\section{Relational Algebra}
  Relational Algebra is a procedural DML
  \subsection{SELECT}
    \subsubsection{definition}
      Selects a subset of the tuples of a relation r, satisfying some condition: $\sigma_B (r) = \{ t \in r | B(t) \}$
    \subsubsection{properties}
      \begin{itemize}
        \item Commutative: $\sigma_{c1}(\sigma_{c2}(R)) = \sigma_{c2}(\sigma_{c1}(R))$
        \item Combination: $\sigma_{c1}(\sigma_{c2}(R)) = \sigma_{c1~AND~c2}(R)$
      \end{itemize}
  \subsection{PROJECT}
    \subsubsection{definition}
      Selects a subset of attributes of a relation. $\pi_X(r) = \{t[X] | t \in r\}$
    \subsubsection{properties}
      \begin{itemize}
        \item if $list1 \subset list2$ then $\pi_{list1}(\pi_{list2}(R)) = \pi_{list1}(R)$, otherwise illegal.
        \item commutative: $\pi_{X}(\sigma_{B}(R)) = \sigma_{B}(\pi_{X}(R))$ if B specifies within X.
      \end{itemize}
  \subsection{UNION}
    Just normal union stuff. $r \cup s = \set{t}{t \in r~OR~t \in s }$
  \subsection{INTERSECTION}
    Just normal intersection stuff. $r \cap s = \set{t}{t \in r~AND~t \in s }$
  \subsection{DIFFERENCE}
    $r - s = \set{t}{t \in r~AND~ t \notin s}$
  \subsection{CARTESIAN PRODUCT}
    $r \times s = \set{t_1 || t_2}{t_1 \in r~AND~t_2 \in s}$\\
    equi-join: join performed based on the equality of matching attributes.
  \subsection{JOIN}
    $r \bowtie_B s = \set{t_1 || t_2}{t_1 \in r~AND~t_2 \in s ~AND~B}$\\
    \subsubsection{Equi-join}
      Equi-join is a theta-join where each comparison operator is “=”.
    \subsubsection{Natural join}
      Natural join is an equi-join where only one attribute from each comparison is retained.
  \subsection{DIVIDE}
    $R \div S = \set{t}{t \times S \in R}$

\section{Database Languages}
  \subsection{SQL}
    \begin{itemize}
      \item Quotes are escaped by doubling them ('')
      \item Some of the frequently-used keywords: ALTER AND CREATE FROM INSERT NOT OR SELECT TABLE WHERE
      \item Built-in datatype: strings, numbers, dates, bit-strings
      \item The NULL value is a member of all data types.
      \item Two kinds of string are available: CHAR(n) - has padding, VARCHAR(n) - no padding
      \item trailing space is ignored in comparison
      \item String
      \begin{itemize}
        \item compare: dict order, LIKE: $\%$ same as *, \_ same as .
        \item manipulation: concat: $||$, LENGTH(s): gives length, SUBSTR(str, start, length) match starting from startth char to the length (start starts from 1)
      \end{itemize}
      \item Dates
      \begin{itemize}
        \item format: DD-Mon-YYYY
        \item (start1, end1) OVERLAPS (start2, end2) -> gives whether two dateranges overlaps
      \end{itemize}
      \item Numbers
      \begin{itemize}
        \item smallint, int, bigint ... 2-bytes, 4-bytes and 8-bytes integers
        \item real, double precision... 4-bytes and 8-bytes floating point
        \item numeric(precision, scale) - The precision of a numeric is the total count of significant digits
in the whole number
        \item $\pi_X(R)$ is implemented in SQL as: SELECT X FROM R
        \item One could eliminate the duplicates by using DISTINCT.
      \end{itemize}
      \item R1 UNION R2, R1 INTERSECT R2, R1 EXCEPT R2
      \item Aggregation: SUM, AVG, MIN, MAX, COUNT
      \item Use HAVING to eliminate groups
    \end{itemize}
  \subsection{PLpgSQL}
    \subsubsection{Create type}
      CREATE TYPE, CREATE DOMAIN \\
      create domain UnswCourseCode as text \\
      check ( value ~ '[A - Z ]{4}[0 -9]{4} ' );
    %TODO: More PLpgSQL
    \subsubsection{Bored let's do it next time}
\section{Functional Dependency}
  \subsection{definition}
    X (functionally) determines Y, $X \rightarrow Y$, iff $t_1[X] = t_2[X] implies t_1[Y] = t_2[Y]$.
  \subsection{infer}
    $F infers X \rightarrow Y ( F |= X \rightarrow Y)$ means $R~satisfies~F -> R~satisfies~X\rightarrow Y$.
  \subsection{Armstrong's axioms}
    \begin{itemize}
      \item Reflexivity: $Y \subseteq X \Rightarrow X \rightarrow Y$
      \item Augmentation $\{X \rightarrow Y\} |= XZ \rightarrow YZ$
      \item Transitivity $\{X \rightarrow Y, Y \rightarrow Z\} |= X \rightarrow Z$
      \item Additivity: $\{X \rightarrow Y, X \rightarrow Z\} |= X \rightarrow YZ$
      \item Projectivity: $\{X \rightarrow YZ\} |= X \rightarrow Y$
      \item Pseudotransitivity $\{X \rightarrow Y, YZ \rightarrow W\} |= XZ \rightarrow W$
    \end{itemize}
  \subsection{Checking FD}
    $F^+$ is the smallest set of FD that contains F and closed under Armstrong's axioms.\\
    Somehow we can prove: $X^+ = \cup_{\forall X \rightarrow A \in F^+}A$ and $X\rightarrow Y \in F^+ \rightleftharpoons Y \subseteq X^+$.
  \subsection{Compute Candidate key}
    Find X such that $X^+ = R$.
    \subsubsection{Algorithm}
      Assign $X$ a super key in $F$ (all attrs on left hand side of FDs). Remove attr from $X$ while remain $X^+ = R$ till no reduction on $X$.
\section{Normal Form}
  \subsection{1NF}
    No multivalue, composite attributes.
    \subsubsection{drawback}
      A lot of duplicated entries
  \subsection{2NF}
    A prime attribute is one that's part of a candidate key. In an FD $X\rightarrow Y$, Y is fully FD on X if no $Z \subset X$ such that $Z \rightarrow Y$, otherwise it's not fulling dependent. 2NF -> all non-prime attr are fully functionally dependent.
    \subsubsection{drawback}
      Record A and record B both associated with something. That might be stored twice.
  \subsection{3NF}
    $X \rightarrow Z$ and $Z \rightarrow Y$ and $Z \nrightarrow X$. 3NF: All NONE\_TRIVIAL FD $X \rightarrow A$ either $X$ is super key or $A$ is prime attr.
  \subsection{BCNF}
    whenever $X \rightarrow A$ holds is non-trivial, $X$ is a super key.
  \subsection{3NF transformation}
    \subsubsection{decomposition}
      $\{R1, R2, R3 .. Rn\}$ is decomposition of R if $\cup^{n}_{i=1}R_i = R$.
    \subsubsection{dependency preserving}
      Two sets F and G are equivalent if $F^+ = G^+$. Decomposition need to make sure: $F^+ = (\cup^n_{i=1}F_i)^+$.
    \subsubsection{testing lossless join}
      1. create S, each $s_{i,j} \in S$ corresponds $R_i$ and attr $A_j$.\\
      2. Repeat:\\
        for each $X \rightarrow Y$, choose rows where all attr in X has a.\\
        In those rows where elements corresponding to Y also take value a if one of the chosen rows take the value a on Y.
    \subsubsection{Algorithm to BCNF}
      while $\exists R \in D~and~R_i \notin BCNF$ \{\\
        find $X \rightarrow Y$ in $R_i$ that violates BCNF, replace $R_i$ in $D$ by $(R_i - Y)~and~(X \cup Y)$;\\
      \}
    \subsubsection{Algorithm to min cover}
      \begin{itemize}
        \item Reduce right: U know how
        \item Reduce right: $X \rightarrow A \in F$ where $X = \set{A_i}{1 <= i <= k}$, if $A \in (X - \{A_i\})^+$ for $i = 1~to~k$.
        \item Remove redundant: U know how
      \end{itemize}
    \subsubsection{3NF decomp algorithm}
      \begin{itemize}
        \item create union of RHS
        \item add all none existing ones into the union.
      \end{itemize}
\section{Disk and File}
  \subsection{storages}
    \begin{itemize}
      \item Primary Storage: main memory
      \item Secondary Storage: hard disk
      \item Tertiary Storage: other stuff
    \end{itemize}
  \subsection{Speed it up}
    \begin{itemize}
      \item Improve Disk Access (Access pattern)
      \item Keep track of free blocks
      \item Use OS FS to manage disk space
    \end{itemize}
  \subsection{Buffer management}
    \subsubsection{Buffer manager}
      manages traffic between disk and memory with buffer pool in main memory. (Buffer pool is collection of frames).
      \begin{itemize}
        \item request\_block: if in buffer, use copy. Otherwise fetch from harddisk then replace.
        \item release\_block: make it good candidate for removal.
        \item write\_block: set dirty bit on
        \item force\_block: force to write to disk
      \end{itemize}
    \subsubsection{Replacement policy}
      \begin{itemize}
        \item LRU
        \item FIFO
        \item MRU
        \item random
      \end{itemize}
      Neither is better
      \begin{itemize}
        \item Fix length record: access by offset
        \item Prefixed with an array of offset
      \end{itemize}
    \subsubsection{Block formats}
      definition: A block is a collection of slots which each contains a record.
\section{Indexing}
  Three types of organisations: Heap Files, Sorted Files, Hashed Files.\\
  D: read/write disk page.\\
  C: process a record.\\
  H: time to hash.\\
  \subsection{Heap}
    B: \# of pages, R: \# of record in a page
    \subsubsection{Scan}
      $B * (D + R * C)$
    \subsubsection{SWES(search with equal selection)}
      $0.5 * B * (D + R * C)$ if on one record, otherwise full scan
    \subsubsection{SWRS}
      $B * (D + R * C)$
    \subsubsection{insert}
      $2 * D + C$(read, process, write)
    \subsubsection{delete}
      if know page, then like write. Otherwise scan + write.
  \subsection{Sorted}
    \subsubsection{Scan}
      $B * (D + R * C)$
    \subsubsection{SWES}
      One record: $log_2(B) * D + C * log_2(R)$,
      multiple: $log_2(B) * D + C * log_2(R) + \#matches$
    \subsubsection{SWRS}
      $log_2(B) * D + C * log_2(R) + \#matches$
    \subsubsection{insert/delete}
      $search + 2 * (0.5 * B * (D + R * C))$
  \subsection{Hashed}
    \subsubsection{Scan}
      $1.25 * B * (D + R * C)$
    \subsubsection{SWES}
      $H + D + 0.5 * C * R$
    \subsubsection{SWRS}
      Scan!
    \subsubsection{Insert/delete}
      If only one blck: $Search + C + D$
  \subsection{Clustered Index}
    A file is organized of data records is the smae as or close to the ordering of data entries in some index.
\section{Transaction}
  \subsection{definition}
    Each program unit is a transaction that either. accesses the contents of the database, or changes the state of the database, from one consistent state to another.
  \subsection{Operations}
    \begin{itemize}
      \item Read
      \item Write
      \item Computation
    \end{itemize}
    Partially committed point: At this point, check and enforce the correctness of the concurrent execution
    Committed state: Once a transaction enters the committed state, it has concluded its execution successfully.
  \subsection{ACID}
    \begin{itemize}
      \item Atomicity: Transaction either performed or not.
      \item Consistency preservation: Correct transaction results in consistent state.
      \item isolation: no updates should be visible until committed
      \item Durability: once committed, change shouldn't be lost.
    \end{itemize}
  \subsection{Recover from failure - Log-based recovery}
      \begin{itemize}
        \item undo logging
        \item redo logging
        \item undo/redo logging
      \end{itemize}
      \begin{itemize}
        \item start transaction \[start transaction, T\]
        \item \[read item, T, X\]
        \item \[write item, T, X, oldV, newV\]
        \item \[commit, T\]
        \item \[abort, T\]
      \end{itemize}
      recovery strategy: undo all uncommitted transactions, redo all committed transactions\\
      If there's checkpoint, only care about those ended after checkpoint.
  \subsection{Concurrency control}
    assumption: a serial schedule is guarranteed to be correct.
    \\ A schedule is serializable if it always produces the same result as some serial schedule.
    \\ Serial schedule: Schedule that does not interleave the actions of different transactions.
    \\ Equivalent schedules: getting same result
    \\ Serializable schedule: result equal to some serial schedule.
    \\ conflict equivalent: 1. involve same actions of the same transactions. 2. conflicting actions are ordered the same way.
    \subsubsection{Testing conflict serializable}
    \begin{itemize}
      \item Construct a schedule graph:
        \begin{itemize}
          \item G = (V, A) consists of vertex set V and arc set A.
          \item V is a transaction. $T_i~and~T_j$, $T_i\rightarrow T_j$ if $O_1 \in T_i, O_2 \in T_j$, $O_1$ before $O_2$, operating on the same item and one of them is write.
        \end{itemize}
      \item check if the graph is cyclic -> non-serializable
    \end{itemize}
    Concurrency control method:
    \begin{enumerate}
      \item Obtain Read lock if $\forall operation, read$ otherwise write lock.
      \item unlock after operation on X in T has been executed.
    \end{enumerate}
    Two phase locking
    \begin{enumerate}
      \item growing phase: all locks for a transaction must be obtained before any locks are released
      \item shrinking phase: gradually release all locks
    \end{enumerate}
    \subsubsection{Deadlock}
      Create a wait-for graph: v for each T, $T_i$ to $T_j$ if $T_i$ is waiting for an item locked by $T_j$. If there's cycle, deadlock!
    \subsubsection{Deadlock solution}
      \begin{enumerate}
        \item deadlock detection: suitable if transactions are short
        \item deadlock prevention: assign priority based on timestamps.
      \end{enumerate}
    \subsubsection{Timestamp ordering}
      Assign timestap to start of transaction.\\
      \lstinputlisting[language=Python]{1.java}
      problems:
      \begin{itemize}
        \item Cyclic restart
        \item cascading rollback
      \end{itemize}
\section{Spatial}
  I have no idea what's going on
\end{document}
